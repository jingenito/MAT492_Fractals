\documentclass{if-beamer}

\title[]{Badass Title}
\subtitle{}
\author{Isaac Ashton, Joseph Ingenito, Peter Tonuzi}
\institute[]{The College of New Jersey}
\date{}
\logo{
\includegraphics[scale=0.10]{tcnj-logotype-400-200.jpg}
}
\subject{MAT 229, Spring 2020} % metadata
\usepackage{animate}

\graphicspath{{Images/}}
\usepackage{movie15}

\begin{document}

\begin{frame}
  \titlepage
\end{frame}

\begin{frame}{What is a fractal string}

\begin{definition}
A {\bf fractal string} $\mathcal{L}$ is a bounded open subset $\Omega$ of the real line.
\end{definition}

\pause
\vspace{.2 in}

$\Omega = \displaystyle\bigcup_{j = 1}^\infty I_j$,\qquad $\displaystyle\text{Vol}_1\left(\Omega\right) = \sum_{j = 1}^\infty \ell_j \implies \lim_{j \to \infty} \ell_j = 0$.

\pause
\vspace{.2 in}

$\ell_1 \geq \ell_2 \geq \ell_3 \geq \cdots \geq 0$

\pause
\vspace{.2 in}

{\color{red} Introduce the Cantor string so that the audience can see an example right away. A movie would be nice!}

\end{frame}

\begin{frame}{Cantor Lawn GIF}
	\begin{center}
		\animategraphics[loop,controls,width=7cm]{10}{Images/CantorLawnPNG/CantorLawnPNG-}{0}{62}
	\end{center}
\end{frame}

\begin{frame}{Dimension of fractal string}

{\bf Inner-tubular neighborhood}:\quad $V(\epsilon) = \text{vol}_1\{x \in \Omega\ |\ d(x,\partial\Omega) < \epsilon\}$

\pause
\vspace{.2 in}

{\color{red} Use a movie of the Cantor String to illustrate the volume formula!}

\pause
\vspace{.2 in}

{\bf Counting function}:\quad $N_{\Omega}(x) = \sum_{\ell_j^{-1} \leq x}$





\begin{enumerate}
\item[] $D_{\mathcal{L}} = \inf\{\alpha \geq 0\ |\ V(\epsilon) = O(\epsilon^{1 - \alpha})\text{ as }\epsilon \to 0^+\}$ \\

\pause

\item[] Lapidus, van Frankenhuijsen, \cite{lapidus2000fractal, lapidus2003complex, lapidus2012fractal}
\[ {\color{blue} D_\mathcal{L} = \inf\{\alpha \in \mathbb{R}\ |\ \sum_{j = 1}^\infty \ell_j^{\alpha} < \infty\} \implies 0 < D_{\mathcal{L}} < 1} \]
\end{enumerate}


\end{frame}


\begin{frame}[allowframebreaks]
\frametitle{References}

\begin{thebibliography}{1}

\bibitem{bini2014solving} 
D. A. Bini and L. Robol.
``Solving secular and polynomial equations: A multiprecision algorithm.'' 
\textit{Journal of Computational and Applied Mathematics}
272
(2014),
276--292.

\bibitem{bremner2011lattice} 
M. R. Bremner.
\textit{Lattice Basis Reduction: An Introduction to the LLL Algorithm and its Applications},
Taylor \& Francis, Boca Raton, 2011.

\bibitem{lapidus1991fractal} 
M. L. Lapidus.
``Fractal drum, inverse spectral problems for elliptic operators and a partial resolution of the Weyl-Berry conjecture.'' 
\textit{Transactions of the American Mathematical Society.}
(2)
325
(1991),
465--529.
 
\bibitem{lapidus1993vibrations} 
M. L. Lapidus.
``Vibrations of Fractal Drums, the Riemann Hypothesis, Waves in Fractal Media and the Weyl-Berry Conjecture.'' In \textit{Ordinary and Partial Differential Equations}, edited by B. D. Sleeman and R. J. Jarvis, Vol. IV, Proc. Twelfth Intern. Conf. (Dundee, Scotland, UK, June 1992), Pitman Research Notes in Math. Series 289, pp. 126--209, London: Longman Scientific and Technical, 1993.

\bibitem{lapidus2008in} 
M. L. Lapidus.
\textit{In Search of the Riemann Zeros: Strings, fractal membranes and noncommutative spacetimes}, 
American Mathematical Society, Providence, Rhode Island, 2008.

\bibitem{lapidus2019an} 
M. L. Lapidus.
``An overview of the complex fractal dimensions: From fractal strings to fractal drums, and back,'' \textit{Contemporary Mathematics}, Vol. 731, Amer. Math. Soc., Providence, R.I., 2019, 143--269

\bibitem{lapidus1993the} 
M. L. Lapidus and C. Pomerance.
``The Riemann Zeta-Function and the One-Dimensional Weyl-Berry Conjecture for Fractal Drums,''
\textit{Proc. London Math. Soc.}
(3)
66
(1993),
41--69.

\bibitem{lapidus1995the} 
M. L. Lapidus and H. Maier.
``The Riemann Hypothesis and Inverse Spectral Problems for Fractal Strings.'' \textit{J. London Math. Soc.}
(2)
52
(1995),
15--34.

\bibitem{lapidus2000fractal} 
M. L. Lapidus and M. van Frankenhuijsen.
\textit{Fractal Geometry and Number Theory \textup(Complex dimensions of fractal strings and zeros of zeta functions\textup)},
Birkh\"auser, Boston, 2000.

\bibitem{lapidus2003complex}
M. L. Lapidus and M. van Frankenhuijsen.
``Complex Dimensions of Self-Similar Fractal Strings and Diophantine Approximation,''  
\textit{J. Experimental Math.}
(1)
12
(2003),
41--69. 

\bibitem{lapidus2012fractal} 
M. L. Lapidus and M. van Frankenhuijsen.
\textit{Fractal Geometry, Complex Dimensions and Zeta Functions: Geometry and Spectra of Fractal Strings}, second edition (of the 2006 edition),
Springer Monographs in Mathematics, Springer, New York, 2012.

\bibitem{lapidus2017fractal} 
M. L. Lapidus and G. Radunovi\'c and D. {\u Z}ubrini\'c.
\textit{Fractal Zeta Functions and Fractal Drums: Higher-Dimensional Theory of Complex Dimensions},
Springer Monographs in Mathematics, Springer, New York, 2017.

\bibitem{lenstra1982factoring}
A. K. Lenstra and H. W. Lenstra and L. Lov\'asz.
``Factoring polynomials with rational coefficients,''  
\textit{Mathematische Annalen}
(4)
261
(1982),
515--534. 

\bibitem{schmidt1980diophantine} 
W. M. Schmidt.
\textit{Diophantine Approximation},
Springer, New York, 1980.

\bibitem{serre1973a} 
J.-P. Serre.
\textit{A Course in Arithmetic\textup{, English translation}},
Springer-Verlag, Berlin, 1973.

\bibitem{voskanian2019on} 
E. K. Voskanian.
``On the Quasiperiodic Structure of the Complex Dimensions of Self-Similar Fractal Strings,''
Ph. D. Dissertation, University of California, Riverside, 2019.

\end{thebibliography}

\end{frame}





\end{document}
